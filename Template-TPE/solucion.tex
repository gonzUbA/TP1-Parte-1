\documentclass[a4paper]{article}

\usepackage{a4wide}
\usepackage{amsmath, amscd, amssymb, amsthm, latexsym}
\usepackage[spanish,activeacute]{babel}
\usepackage{enumerate}
\usepackage[utf8]{inputenc}
\usepackage{amsmath}
\newcommand{\tupla}[2]{#1 \times #2}
\input{Algo1Macros}

\setlength{\parskip}{0.1em}
\usepackage{caratula} % Version modificada para usar las macros de algo1 de ~> https://github.com/bcardiff/dc-tex



\begin{document}
\newcommand{\toroide}{\textit{toroide}}
\newcommand{\bin}{\ensuremath{\ent \times \ent}}

\titulo{TP de Especificación}
\fecha{\today}
\materia{Algoritmos y Estructuras de Datos I}
\grupo{Grupo: 7}

% CAMBIAR INTEGRANTES
\integrante{Teplizky, Gonzalo}{201/20}{gteplizky@dc.uba.ar}
\integrante{Monteys, Lautaro}{5/20}{lmonteys@dc.uba.ar}
\integrante{Stemberg, Uriel}{213/20}{ustemberg@dc.uba.ar}
\integrante{Gómez, Fabián}{522/20}{flaguna@dc.uba.ar}

\maketitle


\section{Ejercicios - Primera Parte}

\aux{Aux}{i: \ent}{\bool}{\True}

\pred{Pred}{s: $\TLista{$\ent$}$, t : \toroide}{\True}


%ESCRIBIR  SOLUCIONES AQUÍ

\begin{ejercicio}[:]
\textbf{pred esValido(t: $\toroide$)}

\pred{esValido}{t: $toroide$}{ \\ (|t| \geq 3 \yLuego |t[0]| \geq 3) \newline \wedge \newline (\forall i : \mathds{Z}) (0\leq i < \longitud{t} \implicaLuego \longitud{t[i]} = \longitud {t[0]}) \\ }
\end{ejercicio}

\begin{ejercicio}[:]
\textbf{pred toroideMuerto(t: $\toroide$)}

\pred{toroideMuerto}{t: $toroide$}{ \\ \hspace{8 cm} (\forall i : \mathds{Z}) (0\leq i < \longitud{t} \implicaLuego \neg  (\exists j: \mathds{Z}) (0 \leq j < \longitud{t[0]} \yLuego t[i][j])) \\ }

\end{ejercicio}


\begin{ejercicio}[:]
\textbf{pred posicionesVivas(t: $\toroide$, vivas : $\TLista{\tupla{\ent}{\ent}}$)}

\pred{posicionesVivas}{t: $\toroide$, vivas : $\TLista{\tupla{\ent}{\ent}}$}{ \\ (\forall x,y : \mathds{Z})(((0\leq x < \longitud{t} \wedge 0\leq y < \longitud{t[0]}) \yLuego t[x][y]) \implicaLuego (x,y) \in vivas) \\ \wedge \\ (\forall v,w : \mathds{Z}) ((v,w)\in vivas \implicaLuego ((0 \leq v <\longitud{t} \wedge 0 \leq w<\longitud{t[0]}) \yLuego t[v][w])) \\ }

\end{ejercicio}



\begin{ejercicio}[:]
\textbf{aux densidadPoblacion(t: $\toroide$) = $\float$}

\auxil{densidadPoblacion(t: toroide) \, : \, $\float$ \enspace}{\newline \frac {(\sum_{i=0}^{|t[0]|-1} \sum_{j=0}^{|t|-1} \IfThenElse{t[i][j]}{1}{0})} {|t|*|t[0]|}}




\end{ejercicio}



\begin{ejercicio}[:]
\textbf{aux cantVecinosVivos(t: $\toroide$, f: $\ent$, c: $\ent$) = $\ent$}

\auxil{cantVecinosVivos(t: toroide, f:\ent, c:\ent) \, : \, $\ent$ \enspace}{\newline(\sum_{i=-1}^1 \sum_{j=-1}^1 \IfThenElse{t[(f+i) \enspace mod \enspace (|t|)]\,[(c+j) \enspace mod \enspace (|t[0]|)]}{1}{0}) - (\IfThenElse{t[f][c]}{1}{0})}

\end{ejercicio}



\begin{ejercicio}[:]
\textbf{pred evolucionDePosicion(t: $\toroide$, posicion : $\tupla{\ent}{\ent}$)}

\pred{evolucionDePosicion}{t: $\toroide$, posicion : $\tupla{\ent}{\ent}$}{\IfThenElse{t[posicion_{0}][posicion_{1}]}{\\ 2 \leq cantVecinosVivos(t, posicion _{0},posicion _{1})\leq 3}{\\ cantVecinosVivos(t, posicion _{0},posicion _{1})= 3} }

\end{ejercicio}



\begin{ejercicio}[:]
\textbf{pred evolucionToroide(t1: $\toroide$, t2: $\toroide$)}

\pred{evolucionToroide}{t1: $\toroide$, t2: $\toroide$}{ \\ (\longitud{t1} = \longitud{t2} \wedge \longitud{t1[0]} = \longitud{t2[0]}) \yLuego  \\(\forall x,y : \mathds{Z}) ((0\leq x < \longitud{t1[0]} \wedge 0\leq y < \longitud{t1})\implicaLuego (t2[x][y] = evolucionDePosicion(t1,(x,y)))) \\}

\end{ejercicio}




\section{Decisiones tomadas}
% Sólo para decisiones de alto nivel, no para explicar la solución a los ejercicios: CONSULTAR!
\paragraph{Consideraciones generales.}
Asumiendo que el toroide es válido, $|t[0]|$ es equivalente  a $ |t[i]| $ para todo $ 0 \leq i < |t|$ ya que cada subsecuencia tiene la misma cantidad de elementos.\newline
	\enspace También, utilizamos la función auxiliar "pertenece" definida en clase para secuencias. La representamos con el símbolo matemático "$\in$"

\paragraph{Ejercicio 1.}
Consideramos un toroide válido a una secuencia de secuencias de booleanos con un mínimo de tres subsecuencias. Cada una de las cuales tenga la misma cantidad de elementos, con un mínimo de tres. 

\paragraph{Ejercicio 4.}
Consideramos que la relación entre la cantidad de casillas vivas y la de totales es la división entre ambas.

\end{document}